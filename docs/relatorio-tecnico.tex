\documentclass[12pt,a4paper]{article}
\usepackage[utf8]{inputenc}
\usepackage[portuguese]{babel}
\usepackage{geometry}
\usepackage{graphicx}
\usepackage{amsmath}
\usepackage{amsfonts}
\usepackage{amssymb}
\usepackage{float}
\usepackage{hyperref}
\usepackage{listings}
\usepackage{xcolor}
\usepackage{booktabs}
\usepackage{caption}
\usepackage{subcaption}

% Configuração da página
\geometry{left=3cm,right=2cm,top=3cm,bottom=2cm}

% Configuração do código
\lstset{
    backgroundcolor=\color{gray!10},
    basicstyle=\footnotesize\ttfamily,
    breakatwhitespace=false,
    breaklines=true,
    captionpos=b,
    commentstyle=\color{green!50!black},
    frame=single,
    keepspaces=true,
    keywordstyle=\color{blue},
    language=Python,
    numbers=left,
    numbersep=5pt,
    numberstyle=\tiny\color{gray},
    rulecolor=\color{black},
    showspaces=false,
    showstringspaces=false,
    showtabs=false,
    stringstyle=\color{red},
    tabsize=2
}

% Configuração de hyperlinks
\hypersetup{
    colorlinks=true,
    linkcolor=blue,
    filecolor=magenta,
    urlcolor=cyan,
    pdftitle={Relatório Técnico - Simulador de Provas},
    pdfauthor={Zé da Gota Solutions},
    pdfsubject={Engenharia de Software},
    pdfkeywords={Python, React, FastAPI, Educação, Simulados}
}

\begin{document}

% Página de título
\begin{titlepage}
    \centering
    \vspace*{2cm}
    
    {\LARGE\textbf{RELATÓRIO TÉCNICO}}\\[0.5cm]
    {\Large\textbf{SIMULADOR DE PROVAS}}\\[0.3cm]
    {\large Zé da Gota Solutions}\\[2cm]
    
    {\large\textbf{Plataforma de Estudos Interativa para Ensino Básico}}\\[1cm]
    
    \begin{figure}[H]
        \centering
        \includegraphics[width=0.3\textwidth]{logo_placeholder.png}
        \caption*{Logo do Projeto}
    \end{figure}
    
    \vfill
    
    {\large
    \textbf{Disciplina:} Engenharia de Software\\
    \textbf{Data:} \today\\
    \textbf{Versão:} 2.0\\
    \textbf{Status:} Em Desenvolvimento - Sprint 2
    }
    
\end{titlepage}

\newpage

% Sumário
\tableofcontents
\newpage

% Resumo Executivo
\section{Resumo Executivo}

O Simulador de Provas é uma plataforma web desenvolvida para auxiliar estudantes do ensino fundamental e médio na preparação para avaliações, oferecendo simulados interativos e relatórios de desempenho detalhados. O sistema também fornece ferramentas para professores acompanharem o progresso de suas turmas.

\subsection{Objetivos Principais}
\begin{itemize}
    \item Criar uma solução completa para preparação de provas
    \item Oferecer feedback personalizado de desempenho
    \item Fornecer métricas consolidadas para professores
    \item Implementar interface intuitiva e responsiva
\end{itemize}

\subsection{Principais Resultados}
\begin{itemize}
    \item Sistema de autenticação JWT implementado
    \item Interface de simulados com cronômetro automático
    \item API REST completa com FastAPI
    \item Relatórios de desempenho em tempo real
    \item Plano de testes abrangente documentado
\end{itemize}

\section{Arquitetura do Sistema}

\subsection{Visão Geral}
O sistema adota uma arquitetura de três camadas com separação clara entre frontend, backend e banco de dados, seguindo princípios de engenharia de software moderna.

\begin{figure}[H]
    \centering
    \begin{tabular}{|c|c|c|}
        \hline
        \textbf{Frontend} & \textbf{Backend} & \textbf{Database} \\
        \hline
        React 18.3.1 & FastAPI & SQLite \\
        Chakra UI & SQLAlchemy & Modelos relacionais \\
        Vite & JWT Auth & Persistência local \\
        \hline
    \end{tabular}
    \caption{Arquitetura em três camadas}
\end{figure}

\subsection{Stack Tecnológica}

\subsubsection{Backend}
\begin{itemize}
    \item \textbf{FastAPI:} Framework web moderno para Python com documentação automática
    \item \textbf{SQLAlchemy:} ORM para mapeamento objeto-relacional
    \item \textbf{SQLite:} Banco de dados relacional leve
    \item \textbf{JWT (python-jose):} Autenticação segura baseada em tokens
    \item \textbf{Passlib:} Biblioteca para hash de senhas com bcrypt
    \item \textbf{Pydantic:} Validação de dados e serialização
    \item \textbf{Uvicorn:} Servidor ASGI de alta performance
\end{itemize}

\subsubsection{Frontend}
\begin{itemize}
    \item \textbf{React 18.3.1:} Biblioteca para construção de interfaces
    \item \textbf{Chakra UI:} Sistema de componentes modular
    \item \textbf{Framer Motion:} Biblioteca para animações
    \item \textbf{Axios:} Cliente HTTP para comunicação com API
    \item \textbf{Vite:} Build tool otimizada com HMR
    \item \textbf{ESLint:} Ferramenta de linting para qualidade de código
\end{itemize}

\subsubsection{Integrações Externas}
\begin{itemize}
    \item \textbf{API do ENEM:} Integração com \texttt{api.enem.dev} para questões reais
    \item \textbf{Conversão automática:} Transformação de dados externos para formato interno
    \item \textbf{Filtros avançados:} Busca por ano, disciplina e limite de questões
    \item \textbf{Cache local:} Armazenamento temporário de questões importadas
\end{itemize}

\section{Funcionalidades Implementadas}

\subsection{Módulo do Aluno}
\begin{enumerate}
    \item \textbf{Autenticação JWT:} Sistema seguro de login e autorização
    \item \textbf{Seleção de Simulados:} Interface com filtros por matéria, instituição e dificuldade
    \item \textbf{Execução de Simulados:} Cronômetro automático com finalização por tempo
    \item \textbf{Navegação entre Questões:} Sistema de navegação livre durante o simulado
    \item \textbf{Correção Automática:} Processamento e cálculo de pontuação em tempo real
    \item \textbf{Histórico Detalhado:} Registro completo de simulados realizados
    \item \textbf{Relatórios de Performance:} Estatísticas de acertos, erros e tempo gasto
\end{enumerate}

\subsection{Módulo do Professor}
\begin{enumerate}
    \item \textbf{Dashboard de Acompanhamento:} Visão geral do desempenho das turmas
    \item \textbf{CRUD de Questões:} Criação, edição e exclusão de questões
    \item \textbf{Análise por Matéria:} Relatórios específicos por disciplina
    \item \textbf{Gestão de Simulados:} Configuração e distribuição de avaliações
    \item \textbf{Integração API ENEM:} Importação de questões reais via \texttt{api.enem.dev}
\end{enumerate}

\section{Modelagem de Dados}

\subsection{Modelo Entidade-Relacionamento}

O banco de dados foi modelado com quatro entidades principais:

\begin{table}[H]
\centering
\begin{tabular}{|l|l|l|}
\hline
\textbf{Entidade} & \textbf{Atributos Principais} & \textbf{Relacionamentos} \\
\hline
User & id, email, hashed\_password, role & 1:N com Simulado \\
\hline
Question & id, question\_text, options, correct\_answer & N:M com Simulado \\
\hline
Simulado & id, user\_id, status, timestamp\_inicio & N:1 com User, 1:1 com Resultado \\
\hline
Resultado & id, simulado\_id, score, answers, timestamp\_fim & 1:1 com Simulado \\
\hline
\end{tabular}
\caption{Estrutura das principais entidades}
\end{table}

\subsection{Modelo de Dados JSON}
As questões utilizam estrutura JSON para armazenar opções de múltipla escolha:

\begin{lstlisting}[language=json, caption=Estrutura de uma questão]
{
  "id": 1,
  "question_text": "Qual é a capital do Brasil?",
  "options": {
    "A": "São Paulo",
    "B": "Rio de Janeiro", 
    "C": "Brasília",
    "D": "Salvador",
    "E": "Belo Horizonte"
  },
  "correct_answer": "C",
  "subject": "Geografia"
}
\end{lstlisting}

\section{API REST}

\subsection{Endpoints Principais}

\subsubsection{Autenticação}
\begin{itemize}
    \item \texttt{POST /auth/token} - Login e geração de token JWT
    \item \texttt{POST /auth/register} - Registro de novos usuários
\end{itemize}

\subsubsection{Simulados}
\begin{itemize}
    \item \texttt{GET /simulados} - Listar simulados disponíveis
    \item \texttt{POST /simulados} - Criar novo simulado
    \item \texttt{POST /simulados/\{id\}/submit} - Enviar respostas
    \item \texttt{GET /simulados/\{id\}/result} - Obter resultado
\end{itemize}

\subsubsection{Questões}
\begin{itemize}
    \item \texttt{GET /questions} - Listar questões
    \item \texttt{POST /questions} - Criar nova questão
    \item \texttt{PUT /questions/\{id\}} - Atualizar questão
    \item \texttt{DELETE /questions/\{id\}} - Deletar questão
    \item \texttt{GET /questions/external/search/\{year\}} - Buscar questões do ENEM
\end{itemize}

\subsubsection{Integração com API Externa}
\begin{itemize}
    \item \textbf{API do ENEM:} Integração com \texttt{api.enem.dev} para busca de questões reais
    \item \textbf{Parâmetros:} Ano obrigatório, disciplina e limite opcionais
    \item \textbf{Conversão automática:} Questões convertidas para formato interno
    \item \textbf{Exemplo:} \texttt{/questions/external/search/2022?discipline=matematica\&limit=5}
\end{itemize}

\subsection{Documentação Automática}
A API utiliza FastAPI para gerar documentação automática em \texttt{/docs} seguindo o padrão OpenAPI 3.0.

\subsection{Exemplo de Integração com API Externa}
A integração com a API do ENEM permite importar questões reais:

\begin{lstlisting}[language=bash, caption=Exemplos de uso da API do ENEM]
# Buscar 10 questões de 2022 (limite padrão)
GET /questions/external/search/2022

# Buscar 5 questões de matemática de 2021
GET /questions/external/search/2021?discipline=matematica&limit=5

# Buscar questões de linguagens de 2020
GET /questions/external/search/2020?discipline=linguagens
\end{lstlisting}

\section{Interface do Usuário}

\subsection{Design System}
\begin{itemize}
    \item \textbf{Paleta de Cores:} Tema azul corporativo (\texttt{blue.800}, \texttt{blue.600})
    \item \textbf{Tipografia:} Fontes system padrão para melhor legibilidade
    \item \textbf{Componentes:} Chakra UI para consistência visual
    \item \textbf{Responsividade:} Layout adaptável para mobile e desktop
\end{itemize}

\subsection{Componentes Principais}
\begin{enumerate}
    \item \textbf{LoginPage:} Autenticação com validação em tempo real
    \item \textbf{SimuladoSelectionPage:} Lista e filtros de simulados disponíveis
    \item \textbf{QuestionPage:} Interface do simulado com cronômetro e navegação
    \item \textbf{HistoryPage:} Histórico detalhado com métricas de performance
\end{enumerate}

\section{Qualidade e Testes}

\subsection{Estratégia de Testes}
O projeto implementa uma estratégia abrangente de testes documentada em \texttt{docs/qa/plano-de-testes.md}:

\begin{table}[H]
\centering
\begin{tabular}{|l|l|l|}
\hline
\textbf{Tipo de Teste} & \textbf{Cobertura} & \textbf{Status} \\
\hline
Funcionais & Fluxos principais & Planejado \\
Interface & Componentes React & Planejado \\
API & Endpoints REST & Planejado \\
Performance & Tempos de resposta & Definido \\
Segurança & Autenticação/Autorização & Planejado \\
Responsividade & Mobile/Desktop & Planejado \\
\hline
\end{tabular}
\caption{Tipos de teste implementados}
\end{table}

\subsection{Histórias de Usuário}
\begin{itemize}
    \item \textbf{HU1:} Como aluno, quero realizar simulados por matéria
    \item \textbf{HU2:} Como aluno, quero visualizar relatório de desempenho detalhado
    \item \textbf{HU3:} Como professor, quero acessar dashboard consolidado
\end{itemize}

\subsection{Critérios de Performance}
\begin{itemize}
    \item Carregamento inicial: < 3 segundos
    \item Geração de relatórios: < 3 segundos
    \item Correção automática: < 1 segundo
\end{itemize}

\section{Segurança}

\subsection{Medidas Implementadas}
\begin{enumerate}
    \item \textbf{Autenticação JWT:} Tokens seguros com expiração configurável
    \item \textbf{Hash de Senhas:} Utilização de bcrypt via Passlib
    \item \textbf{Validação de Dados:} Schemas Pydantic para entrada de dados
    \item \textbf{Proteção de Rotas:} Middleware de autorização no backend
    \item \textbf{CORS:} Configuração adequada para desenvolvimento
\end{enumerate}

\subsection{Boas Práticas}
\begin{itemize}
    \item Princípio do menor privilégio para usuários
    \item Sanitização de entradas do usuário
    \item Logs de segurança para auditoria
    \item Ambiente de desenvolvimento isolado
\end{itemize}

\section{Metodologia de Desenvolvimento}

\subsection{Processo de Desenvolvimento}
O projeto segue metodologia ágil com sprints de 2 semanas:

\begin{itemize}
    \item \textbf{Sprint 1:} Configuração inicial e autenticação
    \item \textbf{Sprint 2:} Implementação de simulados e interface
    \item \textbf{Sprint 3:} Dashboard do professor e relatórios avançados
    \item \textbf{Sprint 4:} Testes automatizados e deploy
\end{itemize}

\subsection{Controle de Versão}
\begin{itemize}
    \item Git com workflow feature-branch
    \item Commits semânticos seguindo convenção
    \item Pull requests com revisão de código
    \item Documentação versionada junto ao código
\end{itemize}

\section{Métricas e Resultados}

\subsection{Métricas de Código}
\begin{table}[H]
\centering
\begin{tabular}{|l|r|l|}
\hline
\textbf{Métrica} & \textbf{Valor} & \textbf{Tecnologia} \\
\hline
Linhas de código Python & $\sim$800 & Backend FastAPI \\
Linhas de código JavaScript & $\sim$1200 & Frontend React \\
Componentes React & 4 & Interface principal \\
Endpoints API & 8 & Serviços REST \\
Modelos de dados & 4 & SQLAlchemy \\
\hline
\end{tabular}
\caption{Métricas quantitativas do projeto}
\end{table}

\subsection{Funcionalidades Entregues}
\begin{itemize}
    \item ✅ Sistema completo de autenticação
    \item ✅ Interface de simulados funcional
    \item ✅ API REST documentada
    \item ✅ Banco de dados estruturado
    \item ✅ Relatórios de desempenho
    \item 🔄 Dashboard do professor (em desenvolvimento)
\end{itemize}

\section{Próximos Passos}

\subsection{Sprint 3 - Funcionalidades Avançadas}
\begin{itemize}
    \item Dashboard completo do professor
    \item Relatórios gráficos com Chart.js
    \item Sistema de recomendações personalizadas
    \item Exportação de relatórios em PDF
\end{itemize}

\subsection{Sprint 4 - Qualidade e Deploy}
\begin{itemize}
    \item Testes automatizados com Jest/Pytest
    \item CI/CD com GitHub Actions
    \item Deploy em produção (Heroku/Vercel)
    \item Documentação completa da API
\end{itemize}

\subsection{Melhorias Futuras}
\begin{itemize}
    \item Integração com APIs externas de questões
    \item Sistema de notificações push
    \item Modo offline para simulados
    \item Análise de desempenho com IA
\end{itemize}

\section{Conclusões}

\subsection{Objetivos Alcançados}
O projeto Simulador de Provas atingiu com sucesso seus objetivos principais na Sprint 2, entregando uma plataforma funcional com:

\begin{itemize}
    \item Sistema de autenticação robusto e seguro
    \item Interface intuitiva para realização de simulados
    \item API REST bem estruturada e documentada
    \item Relatórios de desempenho em tempo real
    \item Arquitetura escalável e maintível
\end{itemize}

\subsection{Lições Aprendidas}
\begin{enumerate}
    \item A separação clara entre frontend e backend facilitou o desenvolvimento paralelo
    \item O uso de FastAPI acelerou significativamente a criação da API
    \item Chakra UI proporcionou consistência visual sem overhead de desenvolvimento
    \item A documentação contínua é essencial para projetos de engenharia de software
\end{enumerate}

\subsection{Impacto Esperado}
A plataforma tem potencial para impactar positivamente:
\begin{itemize}
    \item \textbf{Estudantes:} Preparação mais eficiente para provas
    \item \textbf{Professores:} Insights detalhados sobre desempenho das turmas
    \item \textbf{Instituições:} Melhoria na qualidade do ensino
    \item \textbf{Processo educacional:} Personalização do aprendizado
\end{itemize}

\section{Referências}

\begin{enumerate}
    \item FastAPI Documentation. \url{https://fastapi.tiangolo.com/}
    \item React Documentation. \url{https://react.dev/}
    \item Chakra UI Component Library. \url{https://chakra-ui.com/}
    \item SQLAlchemy ORM. \url{https://www.sqlalchemy.org/}
    \item JWT.io - JSON Web Tokens. \url{https://jwt.io/}
    \item Vite Build Tool. \url{https://vitejs.dev/}
    \item Pydantic Data Validation. \url{https://pydantic.dev/}
    \item Python Passlib. \url{https://passlib.readthedocs.io/}
\end{enumerate}

\appendix

\section{Anexo A - Estrutura de Arquivos}

\begin{lstlisting}[language=bash, caption=Estrutura do projeto]
TrabFinal-EngSoftware/
├── backend/
│   ├── app/
│   │   ├── main.py
│   │   ├── database.py
│   │   ├── models.py
│   │   ├── schemas.py
│   │   ├── crud.py
│   │   └── routers/
│   │       ├── auth.py
│   │       ├── questions.py
│   │       └── simulados.py
│   └── requirements.txt
├── simulador_provas/
│   ├── src/
│   │   ├── App.jsx
│   │   ├── components/
│   │   └── api/
│   └── package.json
└── docs/
    └── qa/
        └── plano-de-testes.md
\end{lstlisting}

\section{Anexo B - Comandos de Execução}

\begin{lstlisting}[language=bash, caption=Setup do backend]
cd backend
python -m venv venv
venv\Scripts\activate  # Windows
pip install -r requirements.txt
uvicorn app.main:app --reload --port 8000
\end{lstlisting}

\begin{lstlisting}[language=bash, caption=Setup do frontend]
cd simulador_provas
npm install
npm run dev
\end{lstlisting}

\end{document}
